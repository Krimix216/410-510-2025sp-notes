\documentclass{lecturenotes}

\usepackage[colorlinks,urlcolor=blue]{hyperref}
\usepackage{doi}
\usepackage{xspace}
\usepackage{agda}
\usepackage{fontspec}
\usepackage{enumerate}
\usepackage{mathpartir}
\usepackage{mathtools}
\usepackage{pl-syntax/pl-syntax}

\newcommand{\abs}[2]{\ensuremath{\lambda #1.\,#2}}
\newcommand{\tabs}[3]{\ensuremath{\lambda #1 \colon #2.\,#3}}
\newcommand{\dbabs}[1]{\ensuremath{\lambda.\,#1}}
\newcommand{\dbind}[1]{\ensuremath{\text{\textasciigrave}#1}}
\newcommand{\app}[2]{\ensuremath{#1\;#2}}
\newcommand{\CC}{\mathbb{C}}

\newcommand{\neutral}[1]{#1\;\text{ne}}
\newcommand{\nf}[1]{#1\;\text{nf}}

\newcommand{\lr}[2]{\ensuremath{\mathsf{P}(#1)(#2)}}
\newcommand{\gdsubst}[2]{\ensuremath{\mathsf{C}(#1)(#2)}}

\newcommand{\proves}{\vdash}

\newenvironment{proof-attempt}{\noindent{\bf Proof Attempt}\hspace*{1em} \textbf{\textcolor{red}{Note: This Proof Fails!}}}{\textbf{\textcolor{red}{Proof Fails.}}}

\title{Logical Relations and Normalization}
\coursenumber{CSE 410/510}
\coursename{Programming Language Theory}
\lecturenumber{23}
\semester{Spring 2025}
\professor{Professor Andrew K. Hirsch}

\begin{document}
\maketitle

\section{Strong Normalization of STLC}

Recall the syntax of the Simply-Typed Lambda Calculus (STLC) without extensions:

\begin{center}
  \begin{syntax}   
    \category[Expressions]{e}
      \alternative{x}
      \alternative{\lambda x:\tau.e}
      \alternative{e_1~e_2}
    \category[Values]{v}
      \alternative{\lambda x:\tau.e}
    \category[Types]{\tau}
      \alternative{A}
      \alternative{\tau_1 \rightarrow \tau_2}
    \category[Typing Contexts]{\Gamma}
      \alternative{\cdot}
      \alternative{\Gamma, x : \tau}
    \category[Evaluation Context]{\CC}
      \alternative{[ ~\cdot~ ]}
      \alternative{\CC ~e}
      \alternative{v~\CC}
  \end{syntax}
\end{center}

We use a call-by-name operational semantics:

\begin{mathpar}
  \inferrule* [left = BetaStep]
    { }
    { (\lambda x.e) v \rightarrow e [x \mapsto v] }

  \inferrule* [left = ContextStep]
    { e_1 \rightarrow e_2}
    { \CC~[ e_1 ] \rightarrow \CC~[ e_2 ] }
\end{mathpar}

As well as its typing rules:

\begin{mathpar}
  \inferrule* [left = Axiom] 
    { x : \tau \in \Gamma}
    { \Gamma \vdash x : \tau}

  \inferrule* [left = $\to$-I]
    { \Gamma, x : \tau_1 \vdash e : \tau_2}
    { \Gamma \vdash \lambda x:\tau_1 .e : \tau_1 \rightarrow \tau_2}

  \inferrule* [left = $\to$-E]
    { 
    \Gamma \vdash e_1 : \tau_1 \rightarrow \tau_2 \\ 
    \Gamma \vdash e_2 : \tau_1
    }
    { \Gamma \vdash e_1 e_2 : \tau_2}
\end{mathpar}

Finally, recall how neutral and normal forms are defined in STLC:

\begin{mathpar}
  \infer{ }{\neutral{x}} \and
  \infer{\neutral{e}}{\nf{e}} \and
  \infer{\nf{e}}{\nf{\abs{x}{e}}} \and
  \infer{\neutral{e_1} \\ \nf{e_2}}{\nf{\app{e_1}{e_2}}}
\end{mathpar}

It is a well-known fact that STLC is \emph{strongly-normalizing}.
That is, that all well-typed terms in STLC eventually $\beta$-reduce to a normal form.

\begin{thm}[Strong Normalization]
  \label{thm:strong-norm}
  $\forall e. \Gamma \proves e : \tau \implies \exists n. \nf{n} \land e \to^\ast n$
\end{thm}

Note that, while Untyped Lambda Calculus does not terminate, well-typed STLC terms do terminate because they reduce to normal forms.
Hence, there is something about the types of STLC that limits our programs to ones that terminate.
Let us now try to prove strong normalization of STLC.

\section{Attempt 1: Induction}

As we have seen, our proofs often proceed by induction, so our first attempt at proving strong normalization is via induction.

\begin{thm}[Strong Normalization]
  \label{thm:strong-norm}
  $\forall e. \Gamma \proves e : \tau \implies \exists n. \nf{n} \land e \to^\ast n$
\end{thm}
\begin{proof-attempt}
  By induction on the evidence for $\Gamma \proves e : \tau$.

  The \textsc{Axiom} and \textsc{$\to$-I} cases are fairly easy, so we will skip them and focus on the \textsc{$\to$-E} case.

  
  \begin{enumerate}[\text{Case}:]
    \item \textsc{$\to$-E}
    
    $e = \app{e_1}{e_2}$ where $\Gamma \proves e_1 : \tau_1 \to \tau_2$ and $\Gamma \proves e_2 : \tau_1$. \\

    IH1: $\exists n_1.~ \nf{n_1} \land e_1 \to^\ast n_1$

    IH2: $\exists n_2.~ \nf{n_2} \land e_2 \to^\ast n_2$

    Using our inductive hypotheses, we can deduce:

    $\app{e_1}{e_2} \to^\ast \app{n_1}{e_2} \to^\ast \app{n_1}{n_2}$

    Since $n_1$ is in normal form, we have two cases.
    First, $n_1$ could be a neutral form, in which case we are done since this means that $\nf{\app{n_1}{n_2}}$.
    Second, $n_1$ could be in normal form, and since it has type $\tau_1 \to \tau_2$ it must have the form $\tabs{x}{\tau_1}{e}$, for some $x$ and $e$.
    So, we can further deduce:

    $\app{\tabs{x}{\tau_1}{e}}{n_2} \to e[x \mapsto n_2]$

    However, at this point, we know nothing about $e$, so we are stuck and the proof cannot proceed further.

  \end{enumerate}

\end{proof-attempt}

\section{Attempt 2: Logical Relation}

Another approach we could take is to use a \emph{logical relation} to prove strong normalization.

\subsection{What is a Logical Relation?}

A logical relation is really a family of recursive relations that are indexed by type.
In Agda, a logical relation \textsf{P} might have the type:

$$\mathsf{P : Type \to Term \to Set}$$

\textsf{P} is a predicate that takes in a type and a term and says that the input term behaves like the input type.

When designing a logical relation, there are three design principles to keep in mind:

\begin{enumerate}[(1)]
  \item Only terms of interest are in the relation.
  \item The property of interest should be baked into the relation.
  \item The relation should be preserved by elimination forms.
\end{enumerate}

If we define a logical relation where our property of interest is strong normalization, the logical relation might look something like this:

\begin{mathpar}
  \lr{A}{e} \triangleq \exists n.~ \nf{n} \land e \to^\ast n \\
  \lr{\tau_1 \to \tau_2}{e} \triangleq (\exists n.~\nf{n} \land e \to^\ast n) \land (\forall e_2.~\lr{\tau_1}{e_2} \implies \lr{\tau_2}{\app{e}{e_2}})
\end{mathpar}

Let us check to make sure that our logical relation follows our design principles.
In our case, all terms are terms of interest, so our logical relation satisfies principle (1).
Note that the property of interest---that $e$ reduces to a normal form---is baked into the logical relation for both types, satisfying principle (2).
Finally, we must show that the relation is preserved by elimination forms.
Expressions of type $A$ have no elimination forms, so we are done for the $A$ case.
Expressions of type $\tau_1 \to \tau_2$ must preserve the relation under function application, which is covered by the second conjunct in \lr{\tau_1 \to \tau_2}{e}.

\subsection{The Fundamental Theorem}

There are two theorems that we are interested in proving using logical relations.
The first theorem says, intuitively, that all well-typed terms are in the logical relation.
The second theorem says, intuitively, that if a term is in the logical relation, then the property of interest is preserved by that term.
Typically, the first theorem, called the \emph{Fundamental Theorem} of the logical relation, is the harder theorem to prove.
The second theorem is typically quite easy to prove, as is the case for our logical relation.
So, our goal will be to prove the Fundamental Theorem for our logical relation.

Before we try to prove the Fundamental Theorem, we must first define what it means to be \emph{semantically well-typed}.
Our logical relation only deals with closed terms---i.e., terms with no free variables---but we can have syntactically well-typed terms that contain free variables.
To deal with this, we need to somehow close the term of interest.
We will leverage parallel substitutions for this---recall that a parallel substitution is simply multiple variable substitutions that occur simultaneously.
We first define what it means to be a ``good (parallel) substitution''.

If $\Gamma$ is a typing context and $\gamma$ is a parallel substitution, then a ``good substitution'' is defined as: \\

\begin{center}
  $\gdsubst{\Gamma}{\gamma} \triangleq$ ``$\gamma$ maps variables to terms that behave like $\Gamma$ says they should''
\end{center}

More formally,

$$\gdsubst{\Gamma}{\gamma} \triangleq \forall x : \tau \in \Gamma.~\lr{\tau}{\gamma(x)}$$

Finally, we define semantic well-typedness, first informally: \\

\begin{center}
  $\Gamma \vDash e : \tau \triangleq$ ``No matter hat I give for $\Gamma$, $e$ acts like a $\tau$''
\end{center}

More formally,

$$\Gamma \vDash e : \tau \triangleq \forall \gamma.~\gdsubst{\Gamma}{\gamma} \implies \lr{\tau}{e[\gamma]}$$

\subsection{Strong Normalization}

We are now ready to prove the Fundamental Theorem for our logical relation.

\begin{thm}[Strong Normalization]
  \label{thm:strong-norm}
  $\forall e. \Gamma \proves e : \tau \implies \exists n. \nf{n} \land e \to^\ast n$
\end{thm}
\begin{proof}
  By induction on the evidence of $\Gamma \proves e : \tau$.

  \item Case: \textsc{Axiom} $\infer{x : \tau \in \Gamma}{\Gamma \proves x : \tau}$
  
  Let $\gamma$ be s.t. $\gdsubst{\Gamma}{\gamma}$. \\
  
  Since we know that $x : \tau \in \Gamma$, then $x[\gamma] = \gamma(x)$.
  Apply $\gdsubst{\Gamma}{\gamma}$ to $x : \tau \in \Gamma$ to obtain $\lr{\tau}{\gamma(x)}$, as desired.

  \item Case: \textsc{$\rightarrow$-I} $\infer{\Gamma, x : \tau_1 \proves e : \tau_2}{\Gamma \proves \tabs{x}{\tau_1 \to \tau_2}{e}}$ \\

  WTS: (1) $\lr{\tau_2}{x[\gamma]}$, (2) $\forall a.~\lr{\tau_1}{a} \implies \lr{\tau_2}{(\tabs{x}{\tau_1}{e})}$ \\

  \begin{enumerate}[(1)]
    \item 
    
    By IH, $\forall \gamma. \gdsubst{\Gamma, x : \tau_1}{\gamma} \implies \lr{\tau_2}{e[\gamma]}$.
    Let $\gamma$ be s.t. $\gdsubst{\Gamma}{\gamma}$.
    Then, $\gamma[x \mapsto x]$ is good: $\gdsubst{\Gamma, x : \tau_1}{\gamma[x \mapsto x]}$.
  
    $$(\tabs{x}{\tau_1}{e})[\gamma] = \tabs{x}{\tau_1}{e[\gamma]} = \tabs{x}{\tau_1}{e[\gamma[x \mapsto x]]}$$
  
    \begin{nb}
    By IH, $e[\gamma]$ normalizes.
    \end{nb}

    $$\tabs{x}{\tau_1}{e[\gamma]} \to^\ast \tabs{x}{\tau_1}{n}\mathsf{,~where }~e[\gamma] \to^\ast n$$.

    \item
    
    Let $a$ be s.t. $\lr{\tau_1}{a}$. \\

    WTS: $\lr{\tau_2}{\tabs{x}{\tau_1}{e[\gamma]}}$ \\

    We will use the following lemma, but we will not prove it.
    \textcolor{blue}{Convince yourself that the following statement is true:}

    $$e \to^\ast e' \land \lr{\tau}{e'} \implies \lr{\tau}{e}$$

    \begin{nb}
      $\gdsubst{\Gamma}{\gamma[x \mapsto a]}$ holds since:

      $$\app{(\tabs{x}{\tau_1}{e[\gamma]})}{a} \to e[\gamma[x \mapsto a]]$$
    \end{nb}

    By IH, $\lr{\tau_2}{\tabs{x}{\tau_1}{e[\gamma[x \mapsto a]]}}$.

  \end{enumerate}

  \item Case: \textsc{$\rightarrow$-E}
  
  \textcolor{blue}{Left as exercise.}

\end{proof}

We can see just by inspecting the logical relation that it all terms in the logical relation are strongly-normalizing.
Since we have proved the Fundamental Theorem, it follows that all syntactically well-typed terms $e$ must be strongly-normalizing.

\end{document}

%%% Local Variables:
%%% mode: latex
%%% TeX-master: t
%%% TeX-engine: luatex
%%% End:
