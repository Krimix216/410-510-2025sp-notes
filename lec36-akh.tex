\documentclass{lecturenotes}

\usepackage[colorlinks,urlcolor=blue]{hyperref}
\usepackage{doi}
\usepackage{xspace}
\usepackage{fontspec}
\usepackage{enumerate}
\usepackage{mathpartir}
\usepackage{pl-syntax/pl-syntax}
\usepackage{forest}
\usepackage{stmaryrd}
\usepackage{epigraph}
\usepackage{xspace}
\usepackage{bbm}
\usepackage{tikz-cd}
\usepackage{unicode-math}

\setsansfont{Fira Code}
\setmathfont{AsanaMath}

\newcommand{\abs}[2]{\ensuremath{\lambda #1.\,#2}}
\newcommand{\tabs}[3]{\ensuremath{\lambda #1 \colon #2.\,#3}}
\newcommand{\dbabs}[1]{\ensuremath{\lambda.\,#1}}
\newcommand{\dbind}[1]{\ensuremath{\text{\textasciigrave}#1}}
\newcommand{\app}[2]{\ensuremath{#1\;#2}}
\newcommand{\utype}{\textsf{unit}\xspace}
\newcommand{\unit}{\ensuremath{\textsf{(}\mkern0.5mu\textsf{)}}}
\newcommand{\prodtype}[2]{\ensuremath{#1 \times #2}}
\newcommand{\pair}[2]{\ensuremath{(#1, #2)}}
\newcommand{\projl}[1]{\ensuremath{\pi_1\mkern2mu#1}}
\newcommand{\projr}[1]{\ensuremath{\pi_2\mkern3mu#1}}
\newcommand{\sumtype}[2]{\ensuremath{#1 + #2}}
\newcommand{\injl}[1]{\ensuremath{\textsf{inj}_1\mkern2mu#1}}
\newcommand{\injr}[1]{\ensuremath{\textsf{inj}_2\mkern3mu#1}}
\newcommand{\case}[5]{\ensuremath{\textsf{case}\mkern5mu#1\mkern5mu\textsf{of}\mkern5mu\injl{#2} \Rightarrow #3;\mkern5mu\injr{#4} \Rightarrow #5\mkern5mu\textsf{end}}}
\newcommand{\vtype}{\textsf{void}\xspace}
\newcommand{\vcase}[1]{\ensuremath{\textsf{case}\mkern5mu#1\mkern5mu\textsf{of}\mkern5mu\textsf{end}}}
\newcommand{\rectype}[2]{\ensuremath{\mu #1.\,#2}}
\newcommand{\roll}[1]{\textsf{roll}\mkern2mu#1}
\newcommand{\unroll}[1]{\textsf{unroll}\mkern2mu#1}
\newcommand{\fatype}[2]{\ensuremath{\forall #1.\,#2}}
\newcommand{\Abs}[2]{\Lambda #1.\,#2}
\newcommand{\App}[2]{#1\;[#2]}
\newcommand{\extype}[2]{\ensuremath{\exists #1.\,#2}}
\newcommand{\pack}[3]{\ensuremath{\textsf{pack}\mkern5mu#1\mathrel{\textsf{as}}#2\mathrel{\textsf{in}}#3}}
\newcommand{\unpack}[4]{\ensuremath{\textsf{unpack}\mkern5mu#1\mathrel{\textsf{as}} #2, #3 \mathrel{\textsf{in}} #4}}
\newcommand{\ltype}[1]{\ensuremath{\textsf{list}\mkern3mu#1}}
\newcommand{\nillist}{\ensuremath{[\,]}}
\newcommand{\conslist}[2]{\ensuremath{#1 \mathop{::} #2}}
\newcommand{\lcase}[5]{\ensuremath{\textsf{case}\mkern5mu#1\mkern5mu\textsf{of}\mkern5mu\nillist\Rightarrow#2; \mkern5mu\conslist{#3}{#4} \Rightarrow #5\mkern5mu\textsf{end}}}
\newcommand{\logn}[1]{\ensuremath{\textsf{log}\mkern3mu#1}}

\newcommand{\capture}[1]{\ensuremath{\lfloor #1 \rfloor}}
\newcommand{\bind}[1]{\ensuremath{\textsf{bind}(#1)}}

\newcommand{\pureeff}{\textsf{pure}\xspace}
\newcommand{\logeff}{\textsf{log}\xspace}

\newcommand{\FV}{\text{FV}}
\newcommand{\BV}{\text{BV}}

\newcommand{\toform}[1]{\ensuremath{\lceil #1 \rceil}}
\newcommand{\totype}[1]{\ensuremath{\lfloor #1 \rfloor}}

\newcommand{\neutral}[1]{#1\;\text{ne}}
\newcommand{\nf}[1]{#1\;\text{nf}}

\newcommand{\subtype}{\ensuremath{\mathrel{\mathord{<}\mathord{:}}}}

\newcommand{\pub}{\text{public}}
\newcommand{\priv}{\text{secret}}

\newcommand{\at}{\ensuremath{\mathrel{@}}}

\newcommand{\obj}[1]{\ensuremath{\mathcal{O}(#1)}}
\renewcommand{\hom}[3][]{\ensuremath{\text{Hom}_{#1}(#2, #3)}}
\newcommand{\id}[1][]{\ensuremath{\mathbbm{1}_{#1}}}

\newcommand{\Set}{\textbf{Set}\xspace}
\newcommand{\Rel}{\textbf{Rel}\xspace}
\newcommand{\Type}{\textbf{Type}\xspace}
\newcommand{\Cat}{\textbf{Cat}\xspace}

\newcommand{\op}[1]{\ensuremath{{#1}^{\text{op}}}}

\newcommand{\prodmor}[2]{\ensuremath{\langle #1, #2 \rangle}}
\newcommand{\inl}{\text{inl}\xspace}
\newcommand{\inr}{\text{inr}\xspace}
\newcommand{\coprodmor}[2]{\ensuremath{[ #1, #2 ]}}

\newcommand{\newchan}[2]{\ensuremath{\nu #1.\,#2}}
\newcommand{\send}[3]{\ensuremath{#1\langle#2\rangle.\,#3}}
\newcommand{\recv}[3]{\ensuremath{#1(#2).\,#3}}

\newcommand{\senda}[2]{\ensuremath{#1\langle#2\rangle}}
\newcommand{\recva}[2]{\ensuremath{#1(#2)}}

\newcommand{\chtype}{\textsf{chan}\xspace}

\title{Concurrent $\lambda$~Calculus 2}
\coursenumber{CSE 410/510}
\coursename{Programming Language Theory}
\lecturenumber{36}
\semester{Spring 2025}
\professor{Professor Andrew K. Hirsch}

\begin{document}
\maketitle

\begin{itemize}
\item Up until now, we have seen a style of concurrency where there are a set number of processes, and processes communicate with one another by name.
\item This is sometimes called CCS-style concurrency, since it is similar to the Calculus of Communicating Systems, a process calculus designed by Robin Milner.
  (This is the same Robin Milner of Hindley-Milner type inference.)
\item Milner went on to describe the $\pi$~calculus, a process calculus that allows processes to spawn and despawn, communicating with one another across \emph{channels}.
\item Today, we will explore this style of communication, again in the form of a concurrent $\lambda$~calculus.
\item We also go back to synchronous communication.
  While synchronicity is an unrealistic assumption, it often makes the theory easier without posing any real threat to the validity of language design, which usually generalizes to asynchronous communication in an ``obvious'' way.
\item We will also see our first type system for a concurrent $\lambda$~calculus, and understand why it does not prevent deadlocks.
  As we will see, this even holds when we prevent ``stupid'' type errors by requiring that all messages be numbers, and thus all messages have the right type.
\end{itemize}

\begin{syntax}
  \category[Types]{\tau} \alternative{\mathbb{N}} \alternative{\chtype} \alternative{\utype} \alternative{\tau_1 \to \tau_2}
  \category[Expressions]{e} \alternative{x} \alternative{\tilde{n}} \alternative{\unit}\\
  \alternative{\tabs{x}{\tau}{e}} \alternative{\app{e_1}{e_2}}\\
  \alternative{\newchan{x}{e}} \alternative{\send{e_1}{e_2}{e_3}} \alternative{\recv{e_1}{x}{e_3}} \alternative{e_1 \parallel e_2}
  \category[Values]{v} \alternative{\tilde{n}} \alternative{\unit} \alternative{\tabs{x}{\tau}{e}}
  \category[Evaluation Contexts]{\mathcal{C}} \alternative{[\cdot]} \alternative{\app{\mathcal{C}}{e}} \alternative{\app{v}{\mathcal{C}}} \alternative{\newchan{x}{e}}\\
  \alternative{\send{\mathcal{C}}{e_2}{e_3}} \alternative{\send{x}{\mathcal{C}}{e}} \alternative{\recv{\mathcal{C}}{y}{e_3}}\\
  \alternative{\mathcal{C} \parallel e} \alternative{e \parallel \mathcal{C}}
  \category[Actions]{\alpha} \alternative{\varepsilon} \alternative{\senda{x}{v}} \alternative{\recva{x}{v}}
\end{syntax}

\begin{itemize}
\item This concurrent $\lambda$ calculus has four additions over STLC with numbers and \utype:
  \begin{enumerate}[(1)]
  \item \textbf{Channel Creation:} $\newchan{x}{e}$ creates a new channel called $x$ that is available in the term $e$.
    Note that there is no such thing as a channel value; channels are just name.
    We can think of this as a name for something in the real world or an outside system, such as a UNIX port or a network controller.
  \item \textbf{Message Sending:} $\send{x}{v}{e}$ sends the message~$v$ on the channel~$x$ and then continues as $e$.
    We require that $v$ is a number~$\tilde{n}$.
  \item \textbf{Message Receipt:} $\recv{x}{y}{e}$ receives a message on the channel~$x$, binds it to the variable~$y$, and then continues as~$e$.
  \item \textbf{Parallel Composition:} $e_1 \parallel e_2$ runs both $e_1$ and $e_2$ at the same time.
    Note that, unlike in the last two systems, parallel composition can appear anywhere inside of an expression, not merely at the top level.
    As such, we can spawn new threads.
    For instance, $\send{x}{\tilde{3}}{(\send{x}{\tilde{4}}{\unit} \parallel\send{y}{\tilde{5}}{\unit})}$ sends $3$ on the channel, then spawns two threads.
    One of these two threads sends $4$ on $x$, while the other sends $5$ on $y$.
    Both threads then end with $\unit$.
    In general, we require that at least one of the spawned threads return $\utype$, so that only one value is returned from an expression.
  \end{enumerate}
\item As before, we give semantics to our concurrent $\lambda$~calculus using a labeled transition system, where the labels represent the communication with the outside world.
\end{itemize}

\begin{mathpar}
  \infer{e \xrightarrow{\alpha} e'}{\mathcal{C}[e] \xrightarrow{\alpha} \mathcal{C}[e']}\and
  \infer{ }{\app{(\tabs{x}{\tau}{e})}{v} \xrightarrow{\varepsilon} e[x \mapsto v]}\\
  \infer{ }{\send{x}{v}{e} \xrightarrow{\senda{x}{v}} e} \and
  \infer{ }{\recv{x}{y}{e} \xrightarrow{\recva{x}{v}} e[y \mapsto v]} \\
  \infer{e_1 \xrightarrow{\senda{x}{v}} e_1'\\ e_2 \xrightarrow{\recva{x}{v}} e_2'}{e_1 \parallel e_2 \xrightarrow{\varepsilon} e_1' \parallel e_2'} \and
  \infer{e_1 \xrightarrow{\recva{x}{v}} e_1'\\ e_2 \xrightarrow{\senda{x}{v}} e_2'}{e_1 \parallel e_2 \xrightarrow{\varepsilon} e_1' \parallel e_2'} \\
  \infer{ }{\unit \parallel e \xrightarrow{\varepsilon} e} \and
  \infer{ }{e \parallel \unit \xrightarrow{\varepsilon} e}\\
    \infer{x \not\in \FV(e)}{\newchan{x}{e} \xrightarrow{\varepsilon} e}\and
  \infer{e \xrightarrow{\alpha} e'\\ \forall v.\, \alpha \neq \recva{x}{v} \land \alpha \neq \senda{x}{v}}{\newchan{x}{e} \xrightarrow{\alpha} \newchan{x}{e'}}\\

\end{mathpar}

\begin{itemize}
\item Like before, the normal $\lambda$~calculus reduction steps generate silent steps labeled with~$\varepsilon$.
\item In addition, we allow a send and a matching receive to both step at once on either side of a parallel composition, creating a silent step.
\item When creating a new channel, there are two possibilities:
  \begin{enumerate}[(1)]
  \item We can get rid of the channel creation if it is never used
  \item If a channel is used, then it allows steps which do \emph{not} involve that channel to percolate up.
  \end{enumerate}
\item Intuitively, these mean that any send or receive on a channel must be matched with another send or receive on that channel under the new channel construct.
  This turns them into a silent step, which is allowed with a new channel construct.
  Once all of the sends and receives on a channel are done, the channel disappears.
\item A parallel composition with unit can take a silent step to get rid of the finished thread.
  Since we enforce that at least one side of a parallel be a unit, this means that only one value is computed by an expression.
\end{itemize}

\begin{mathpar}
  \infer{x : \tau \in \Gamma}{\Gamma \vdash x : \tau}\and
  \infer{ }{\Gamma \vdash \tilde{n} : \mathbb{N}} \and
  \infer{ }{\Gamma \vdash \unit : \utype}\\
  \infer{\Gamma, x : \tau_1 \vdash e: \tau_2}{\Gamma \vdash \tabs{x}{\tau_1}{e} : \tau_1 \to \tau_2}\and
  \infer{\Gamma \vdash f : \tau_1 \to \tau_2\\ \Gamma \vdash a : \tau_1}{\Gamma \vdash \app{f}{a} : \tau_2} \\
  \infer{\Gamma, x : \chtype \vdash e: \tau}{\Gamma \vdash \newchan{x}{e} : \tau}\and
  \infer{\Gamma \vdash e_1 : \chtype\\ \Gamma \vdash e_2 : \mathbb{N}\\ \Gamma \vdash e_3 : \tau}{\Gamma \vdash \send{e_1}{e_2}{e_3} : \tau}\and
  \infer{\Gamma \vdash e_1 : \chtype\\\Gamma, x : \mathbb{N} \vdash e_2 : \tau}{\Gamma \vdash \recv{e_1}{x}{e_2} : \tau}\\
  \infer{\Gamma \vdash e_1 : \utype\\ \Gamma \vdash e_2 : \tau}{\Gamma \vdash e_1 \parallel e_2 : \tau}\and
  \infer{\Gamma \vdash e_1 : \tau\\ \Gamma \vdash e_2 : \utype}{\Gamma \vdash e_1 \parallel e_2 : \tau}\and
\end{mathpar}

\begin{itemize}
\item We can give a data type system to our concurrent $\lambda$~calculus, as normal.
\item Channel creation merely binds the variable $x$, marking it as denoting a channel.
\item In order to send a message, we must be sending a number on a channel.
 That way, when receiving a message we know it must be a number.
\item Finally, we enforce that at least one side of a parallel be a unit.
\item We get the following theorem:
  \begin{thm}[Preservation]
    If $\Gamma \vdash e : \tau$ and $e \xrightarrow{\alpha} e'$, then $\Gamma \vdash e' : \tau$.
  \end{thm}
\item However, we do \emph{not} get any version of progress.
\item In particular, we get the following:
  $$\vdash \newchan{x}{(\recv{x}{y}{\send{x}{y}{\unit}} \parallel \recv{x}{y}{\send{x}{y}{\unit}})} : \utype$$
  Nevertheless, this program deadlocks.
\end{itemize}

\end{document}

%%% Local Variables:
%%% mode: latex
%%% TeX-master: t
%%% TeX-engine: luatex
%%% End:
