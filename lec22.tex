\documentclass{lecturenotes}

\usepackage[colorlinks,urlcolor=blue]{hyperref}
\usepackage{doi}
\usepackage{xspace}
\usepackage{fontspec}
\usepackage{enumerate}
\usepackage{mathpartir}
\usepackage{pl-syntax/pl-syntax}
\usepackage{forest}
\usepackage{stmaryrd}
\usepackage{epigraph}
\usepackage{xspace}

\setsansfont{Fira Code}

\newcommand{\abs}[2]{\ensuremath{\lambda #1.\,#2}}
\newcommand{\tabs}[3]{\ensuremath{\lambda #1 \colon #2.\,#3}}
\newcommand{\dbabs}[1]{\ensuremath{\lambda.\,#1}}
\newcommand{\dbind}[1]{\ensuremath{\text{\textasciigrave}#1}}
\newcommand{\app}[2]{\ensuremath{#1\;#2}}
\newcommand{\utype}{\textsf{unit}\xspace}
\newcommand{\unit}{\ensuremath{\textsf{(}\mkern0.5mu\textsf{)}}}
\newcommand{\prodtype}[2]{\ensuremath{#1 \times #2}}
\newcommand{\pair}[2]{\ensuremath{(#1, #2)}}
\newcommand{\projl}[1]{\ensuremath{\pi_1\mkern2mu#1}}
\newcommand{\projr}[1]{\ensuremath{\pi_2\mkern3mu#1}}
\newcommand{\sumtype}[2]{\ensuremath{#1 + #2}}
\newcommand{\injl}[1]{\ensuremath{\textsf{inj}_1\mkern2mu#1}}
\newcommand{\injr}[1]{\ensuremath{\textsf{inj}_2\mkern3mu#1}}
\newcommand{\case}[5]{\ensuremath{\textsf{case}\mkern5mu#1\mkern5mu\textsf{of}\mkern5mu\injl{#2} \Rightarrow #3;\mkern5mu\injr{#4} \Rightarrow #5\mkern5mu\textsf{end}}}
\newcommand{\vtype}{\textsf{void}\xspace}
\newcommand{\vcase}[1]{\ensuremath{\textsf{case}\mkern5mu#1\mkern5mu\textsf{of}\mkern5mu\textsf{end}}}
\newcommand{\rectype}[2]{\ensuremath{\mu #1.\,#2}}
\newcommand{\roll}[1]{\textsf{roll}\mkern2mu#1}
\newcommand{\unroll}[1]{\textsf{unroll}\mkern2mu#1}
\newcommand{\fatype}[2]{\ensuremath{\forall #1.\,#2}}
\newcommand{\Abs}[2]{\Lambda #1.\,#2}
\newcommand{\App}[2]{#1\;[#2]}
\newcommand{\extype}[2]{\ensuremath{\exists #1.\,#2}}
\newcommand{\pack}[3]{\ensuremath{\textsf{pack}\mkern5mu#1\mathrel{\textsf{as}}#2\mathrel{\textsf{in}}#3}}
\newcommand{\unpack}[4]{\ensuremath{\textsf{unpack}\mkern5mu#1\mathrel{\textsf{as}} #2, #3 \mathrel{\textsf{in}} #4}}

\newcommand{\FV}{\text{FV}}
\newcommand{\BV}{\text{BV}}

\newcommand{\toform}[1]{\ensuremath{\lceil #1 \rceil}}
\newcommand{\totype}[1]{\ensuremath{\lfloor #1 \rfloor}}

\newcommand{\neutral}[1]{#1\;\text{ne}}
\newcommand{\nf}[1]{#1\;\text{nf}}

\newcommand{\subtype}{\ensuremath{\mathrel{\mathord{<}\mathord{:}}}}

\title{Existential types}
\coursenumber{CSE 410/510}
\coursename{Programming Language Theory}
\lecturenumber{22}
\semester{Spring 2025}
\professor{Professor Andrew K. Hirsch}

\begin{document}
\maketitle

\section{Existential Types}
\label{sec:existential types}

\begin{itemize}
  \item Previously we saw $\forall$ types, or universal types. Now we look existential types, or $\exists$ types.
  \item So far, we have learned that parametric polymorphism allowed a program to be given generic types, allowing it to be instantiated with different types in different situations.
  \item But parametric polymorphism also allows us to essentially conceal the actual type by abstracting over it.
  \item For example, an OCaml module can be seen as an existential type as it can define a type, but not tell you what it is. You can still use the functions in the module, even if you don’t know the hidden type.
\end{itemize}

\begin{syntax}
  \category[Types]{\tau} \alternative{\ldots} \alternative{\extype{X}{\tau}}
  \category[Expressions]{e} \alternative{\dots} \alternative{\pack{\tau_1}{X}{e}} \alternative{\unpack{e_1}{X}{x}{e_2}}
  \category[Evaluation Contexts]{\mathcal{C}} \alternative{\ldots} \alternative{\pack{\tau_1}{X}{C}} \alternative{\unpack{\mathcal{C}}{X}{x}{e_2}}
\end{syntax}

\begin{itemize}
\item $\extype{X}{\tau}$ means ``a $\tau$ where the definition of the type $X$ is hidden.''
\item $\pack{\tau}{X}{e}$ is how we create $\extype{X}{\tau}$, and it means to choose $\tau$ as the definition of X, and $e$ as the value. 
It also means that we know about $\tau$ in $e$, but not outside of $e$.
\item $\unpack{e_1}{X}{x}{e_2}$ is how we use existential types, and it means we take a hidden type and value from $e_1$, name them $X$ and $x$,
and use them inside $e_2$ without knowing what $X$ actually is.
\item Because we don't know what $X$'s type is in \textsf{unpack}, $e_2$ can't use definition of $X$ inside of $e_1$.
\end{itemize}

\begin{mathpar}
  \infer*[left=$\exists$-I]{\Delta; \Gamma \vdash e : \tau_2[X \mapsto \tau_1]}{\Delta; \Gamma \vdash \pack{\tau_1}{X}{e} : \extype{X}{\tau_2}} \and
  \infer*[right=$\exists$-E]{\Delta; \Gamma \vdash e_1 : \extype{X}{\tau_1}\\ \Delta, X; \Gamma, x : \tau_1 \vdash e_2 : \tau_2\\\Delta \vdash \tau_2}{\Delta; \Gamma \vdash \unpack{e_1}{X}{x}{e_2} : \tau_2}
\end{mathpar}

\begin{itemize}
\item In \textsc{$\exists$-E}, the premise $\Delta \vdash \tau_2$ is indicating that the only free type variable for $\tau_2$ is in $\Delta$.
\item Therefore this means that $X$ can't be containing $\tau_2$.
\item This shows exactly how existential types hide information: the return type of an \textsf{unpack} must not depend on the specific type $X$ from $e_1$, 
so that the outside world can't see what type $X$ really is. \newline \newline
\end{itemize}

\begin{itemize}
  \item Here we see the $\beta$ and $\eta$ rules for existential types.
\end{itemize}
\begin{mathpar}
  \unpack{(\pack{\tau_1}{X}{v})}{X}{x}{e_2} \to e_2[X \mapsto \tau_1, x \mapsto v] \and
  \unpack{e}{X}{x}{\left(\pack{X}{X}{x}\right)} \equiv_\eta e
\end{mathpar}

\newpage
\section{Example: The Counter}
\label{sec:counter}
\begin{itemize}
\item The counter can only tick up or reset. 
\item Here we see \textsf{counter}, an example of an existential type:
  $$\textsf{counter} \triangleq \extype{X}{\{\textsf{init} : \utype \to X; \textsf{getCount} : X \to \mathbb{N}; \textsf{tick} : X \to X; \textsf{reset} : X \to X\}}$$
\item And here we see \textsf{numCounter}, an example of how to use that existential type:
  $$\textsf{numCounter} \triangleq \pack{\mathbb{N}}{X}{\{\textsf{init} = \abs{x}{0}; \textsf{getCount} = \abs{x}{x}; \textsf{tick} = \abs{x}{x + 1}; \textsf{reset} = \abs{x}{0}\}} : \textsf{counter}$$
\item \textsf{counter} is an existential type, and it has 4 functions: \textsf{init}, \textsf{getCount}, \textsf{tick}, and \textsf{reset}.
\item And \textsf{numCounter} is an implementation of the counter as a natural number:
    \begin{itemize}
    \item \textsf{init} initialises, and so it returns 0 no matter what its input is
    \item \textsf{getCount} returns the current number, which is just the identity 
    \item \textsf{tick} just adds one to the number
    \item \textsf{reset} just resets and so it also returns 0 no matter what its input is
    \end{itemize}
\item Here we see an \emph{almost} correct (but still wrong) example:
  $$
  \begin{array}{l}
    \lambda \textsf{ctr} \mathrel{:} \textsf{counter}.\\
    \begin{array}{l}
      \textsf{unpack}\mkern5mu\textsf{ctr} \mathrel{\textsf{as}} X,\, \textsf{impl}\;\textsf{in}\\
      \begin{array}{l}
        \textsf{let}\mkern5mux = \app{\textsf{impl}.\textsf{init}}{\unit}\;\textsf{in}\\
        \begin{array}{l}
          \textsf{let}\mkern5mux' = \app{\textsf{impl}.\textsf{tick}}{x}\;\textsf{in}\\
          \begin{array}{l}
            (\app{\textsf{impl}.\textsf{getCount}}{x'}, \app{\textsf{impl}.\textsf{reset}}{x'})
          \end{array}
        \end{array}
      \end{array}
    \end{array}
  \end{array}
  $$
\item The type of this function would be \textsf{counter} $\to \mathbb{N} * X$.
\item The X is where the issue is. It should never be leaking any information about how $X$ is implemented.
\item As it was leaking information, this example was \emph{almost} correct but is still wrong. \newline
\item We now see the correct version:
  $$
  \begin{array}{l}
    \lambda \textsf{ctr} \mathrel{:} \textsf{counter}.\\
    \begin{array}{l}
      \textsf{unpack}\mkern5mu\textsf{ctr} \mathrel{\textsf{as}} X,\, \textsf{impl}\;\textsf{in}\\
      \begin{array}{l}
        \textsf{let}\mkern5mux = \app{\textsf{impl}.\textsf{init}}{\unit}\;\textsf{in}\\
        \begin{array}{l}
          \textsf{let}\mkern5mux' = \app{\textsf{impl}.\textsf{tick}}{x}\;\textsf{in}\\
          \begin{array}{l}
            (\app{\textsf{impl}.\textsf{getCount}}{x'}, \app{\textsf{impl}.\textsf{getCount}}{(\app{\textsf{impl}.\textsf{reset}}{x'})})
          \end{array}
        \end{array}
      \end{array}
    \end{array}
  \end{array}
  $$  
\item The type of this function would be \textsf{counter} $\to \mathbb{N} * \mathbb{N}$.
\item Which means it doesn't leak any information about $X$, as it correctly shouldn't be. 
\item Since the function doesn’t know what the hidden type $X$ actually is, the only things it can do with values of type $X$ are the operations in the record.
\item This is an example where existential types are useful, where we can define abstract types with only certain operations that are allowed.
\end{itemize}

\section{Encoding Existential Types}
\label{sec:encod-exist-types}

\begin{itemize}
\item Just like with other types, we can represent existential types using only universal types.
\item A program using an existential type should work with any possible concrete type that might be hidden inside, as long as it follows the provided operations.
\item So we can encode existential types into System F like this:
  $$
  \begin{array}{rl}
    \extype{X}{\tau} \triangleq& \fatype{Y}{(\fatype{X}{\tau} \to Y) \to Y}\\
    \pack{\tau_1}{X}{e} \triangleq& \Abs{Y}{\tabs{f}{\fatype{X}{\tau \to Y}}{\app{\App{f}{\tau_1}}{e}}}\\
    \unpack{e_1}{X}{x}{e_2} \triangleq& \app{\App{e_1}{\tau_2}}{(\Abs{X}{\tabs{x}{\tau}{e_2}})}
  \end{array}
  $$
\item And we can check with the $\beta$ law:
  $$
  \begin{array}{rl}
    \unpack{(\pack{\tau_1}{X}{v})}{X}{x}{e_2} =& \app{\App{(\Abs{Y}{\tabs{f}{\fatype{X}{\tau \to Y}}{\app{\App{f}{\tau_1}}{v}}})}{\tau_2}}{(\Abs{X}{\tabs{x}{\tau}{e_2}})}\\
    \to& \app{(\tabs{f}{\fatype{X}{\tau \to \tau_2}}{\app{\App{f}{\tau_1}}{v}})}{(\Abs{X}{\tabs{x}{\tau}{e}})}\\
    \to& \app{\App{(\Abs{X}{\tabs{x}{\tau_1}{e}})}{\tau}}{v}\\
    \to& \app{(\tabs{x}{\tau_1[X\mapsto\tau]}{e[X \mapsto \tau}])}{v}\\
    \to& e[X \mapsto \tau, x \mapsto v]
  \end{array}
  $$
\end{itemize}

\end{document}

%%% Local Variables:
%%% mode: latex
%%% TeX-master: t
%%% TeX-engine: luatex
%%% End:
