\documentclass{lecturenotes}

\usepackage[colorlinks,urlcolor=blue]{hyperref}
\usepackage{doi}
\usepackage{xspace}
\usepackage{fontspec}
\usepackage{enumerate}
\usepackage{mathpartir}
\usepackage{pl-syntax/pl-syntax}
\usepackage{forest}
\usepackage{stmaryrd}
\usepackage{epigraph}
\usepackage{xspace}
\usepackage{bbm}

\setsansfont{Fira Code}

\newcommand{\abs}[2]{\ensuremath{\lambda #1.\,#2}}
\newcommand{\tabs}[3]{\ensuremath{\lambda #1 \colon #2.\,#3}}
\newcommand{\dbabs}[1]{\ensuremath{\lambda.\,#1}}
\newcommand{\dbind}[1]{\ensuremath{\text{\textasciigrave}#1}}
\newcommand{\app}[2]{\ensuremath{#1\;#2}}
\newcommand{\utype}{\textsf{unit}\xspace}
\newcommand{\unit}{\ensuremath{\textsf{(}\mkern0.5mu\textsf{)}}}
\newcommand{\prodtype}[2]{\ensuremath{#1 \times #2}}
\newcommand{\pair}[2]{\ensuremath{(#1, #2)}}
\newcommand{\projl}[1]{\ensuremath{\pi_1\mkern2mu#1}}
\newcommand{\projr}[1]{\ensuremath{\pi_2\mkern3mu#1}}
\newcommand{\sumtype}[2]{\ensuremath{#1 + #2}}
\newcommand{\injl}[1]{\ensuremath{\textsf{inj}_1\mkern2mu#1}}
\newcommand{\injr}[1]{\ensuremath{\textsf{inj}_2\mkern3mu#1}}
\newcommand{\case}[5]{\ensuremath{\textsf{case}\mkern5mu#1\mkern5mu\textsf{of}\mkern5mu\injl{#2} \Rightarrow #3;\mkern5mu\injr{#4} \Rightarrow #5\mkern5mu\textsf{end}}}
\newcommand{\vtype}{\textsf{void}\xspace}
\newcommand{\vcase}[1]{\ensuremath{\textsf{case}\mkern5mu#1\mkern5mu\textsf{of}\mkern5mu\textsf{end}}}
\newcommand{\rectype}[2]{\ensuremath{\mu #1.\,#2}}
\newcommand{\roll}[1]{\textsf{roll}\mkern2mu#1}
\newcommand{\unroll}[1]{\textsf{unroll}\mkern2mu#1}
\newcommand{\fatype}[2]{\ensuremath{\forall #1.\,#2}}
\newcommand{\Abs}[2]{\Lambda #1.\,#2}
\newcommand{\App}[2]{#1\;[#2]}
\newcommand{\extype}[2]{\ensuremath{\exists #1.\,#2}}
\newcommand{\pack}[3]{\ensuremath{\textsf{pack}\mkern5mu#1\mathrel{\textsf{as}}#2\mathrel{\textsf{in}}#3}}
\newcommand{\unpack}[4]{\ensuremath{\textsf{unpack}\mkern5mu#1\mathrel{\textsf{as}} #2, #3 \mathrel{\textsf{in}} #4}}

\newcommand{\FV}{\text{FV}}
\newcommand{\BV}{\text{BV}}

\newcommand{\toform}[1]{\ensuremath{\lceil #1 \rceil}}
\newcommand{\totype}[1]{\ensuremath{\lfloor #1 \rfloor}}

\newcommand{\neutral}[1]{#1\;\text{ne}}
\newcommand{\nf}[1]{#1\;\text{nf}}

\newcommand{\subtype}{\ensuremath{\mathrel{\mathord{<}\mathord{:}}}}

\newcommand{\pub}{\text{public}}
\newcommand{\priv}{\text{secret}}

\newcommand{\at}{\ensuremath{\mathrel{@}}}

\newcommand{\obj}[1]{\ensuremath{\mathcal{O}(#1)}}
\renewcommand{\hom}[3][]{\ensuremath{\text{Hom}_{#1}(#2, #3)}}
\newcommand{\id}[1][]{\ensuremath{\mathbbm{1}_{#1}}}

\newcommand{\Set}{\textbf{Set}\xspace}
\newcommand{\Rel}{\textbf{Rel}\xspace}
\newcommand{\Type}{\textbf{Type}\xspace}

\newcommand{\op}[1]{\ensuremath{{#1}^{\text{op}}}}

\title{Categories}
\coursenumber{CSE 410/510}
\coursename{Programming Language Theory}
\lecturenumber{29}
\semester{Spring 2025}
\professor{Professor Andrew K. Hirsch}

\begin{document}
\maketitle

\begin{itemize}
\item We now turn to our second form of denotational semantics: categories.
\item This is more traditionally viewed as a form of denotational semantics than logical relations are.
  \begin{itemize}
  \item Traditionally, logical relations were seen as a proof method rather than a denotational semantics
  \item However, this view is changing
  \item There are some purists who will not refer to either logical relations or categories as denotational semantics, but they're unusual.
  \end{itemize}
\item Most of the time we spend on category theory will be spent on definitions.
  This is largely just the way learning category theory goes; we will see some minor theorems and examples.
  It's important not to get too overwhelmed with definitions!
\item In this lecture, we focus on the idea of a category, see some examples, and start to understand why it's useful in semantics.
\end{itemize}

\section{Categories, Defined}
\label{sec:categories-defined}

\begin{itemize}
\item Category theory is the mathematical study of compositionality.
\item As such categories are an algebraic object with composition as its main operation.
\item If this is the first time you've seen an algebraic definition, think of it like a record: anything that has \emph{all} of the operations and properties defined is a category.
\end{itemize}

\begin{defn}[Category]
  A \emph{category}~$\mathcal{C}$ consists of:
  \begin{itemize}
  \item A collection $\obj{\mathcal{C}}$ of objects.
  \item For every pair $X, Y \in \obj{\mathcal{C}}$, a collection $\hom[\mathcal{C}]{X}{Y}$ of \emph{morphisms between $X$ and $Y$}.
    When the category is obvious from context, we will just write $\hom{X}{Y}$.
    We also often write $f : X \to Y$ for $f \in \hom{X}{Y}$.
    We say that $X$ is the \emph{domain} of $X$, while $Y$ is its $\emph{codomain}$.
  \item For every object $X \in \obj{C}$, a morphism~$\id[X]$ called the \emph{identity} at $X$.
    If the object is obvious from context, we will often just write $\id$.
  \item For every triple of objects a \emph{composition} operator $; : \hom{X}{Y} \to \hom{Y}{Z} \to \hom{X}{Z}$.
    In other words, if $f : X \to Y$ and $g : Y \to Z$, then there is a morphism $f; g : X \to Z$.
    We sometimes write $g \circ f$ instead; this is common in textbook definitions of categories.
    These are referred to as ``diagram-order'' and ``function-order'' composition, respectively.
  \end{itemize}
  Obeying the following laws:
  \begin{itemize}
  \item The identity acts like an identity for composition on both the left and right:
    $$\id[X]; f = f = f; \id[Y] \;\text{where}\,f : X \to Y$$
  \item Composition is associative: if $f : W \to X$, $g : X \to Y$, and $h : Y \to Z$, then
    $$ f; (g; h) = (f; g); h$$
  \end{itemize}
\end{defn}

\section{Examples of Categories in Mathematics}
\label{sec:math-examples-categories}

Let's look at some examples of categories that you might know from mathematics.

\subsection{Sets}
\label{sec:sets}

The category \Set consists of:
\begin{itemize}
\item \textbf{Objects}: All sets
\item \textbf{Morphisms}: $\hom[\Set]{X}{Y} = Y^X$ the set of functions from $X$ to $Y$.
  Recall that in sets a function from $X$ to $Y$ is a set $f \subseteq X \times Y$ such that $\forall x \in X, \exists ! y \in Y.\, (x, y) \in f$, and that we write $f(x)$ for the unique $y$ such that $(x, y) \in f$.
\item \textbf{Identities}: $\id[X] = \{(x, x) \mid x \in X\}$.
\item \textbf{Composition}: if $f : X \to Y$ and $g : Y \to Z$, then $f; g = \{ (x, z) \mid \exists y.\, (x, y) \in f \land (y, z) \in g\}$.
\end{itemize}

We need to show that these satisfy the category laws.
First, note that
$$f; \id[Y] = \{(x, y_2) \mid \exists y_1.\, (x, y_1) \in f \land (y_1, y_2) \in \id[Y]\} = \{(x, y) \mid (x, y) \in f\} = f$$
Second,
$$\id[X]; f = \{(x_1, y) \mid \exists x_2.\, (x_1, x_2) \in id[X] \land (x_2, y) \in f\} = \{(x, y) \mid (x, y) \in f\} = f$$

Finally,
$$
\begin{array}{ll}
  f; (g; h) &= \{(w, z) \mid \exists x.\, (w, x) \in f \land (x, z) \in g; h\}\\
            &= \{(w, z) \mid \exists x.\, (w, x) \in f \land \exists y.\, (x, y) \in g \land (y, z) \in h\}\\
            &= \{(w, x) \mid \exists x, y. (w, x) \in f \land(x, y) \in g \land (y, z) \in h\}\\
            &= \{(w, x) \mid \exists y. (\exists x. (w, x) \in f \land(x, y) \in g) \land (y, z) \in h\}\\
            &=\{(w, x) \mid \exists y.\, (w, y) \in f; g \land (y, z) \in h\}\\
            &=(f; g); h
\end{array}
$$

\subsection{Relations}
\label{sec:relations}

The category \Rel consists of:
\begin{itemize}
\item \textbf{Objects}: All sets
\item \textbf{Morphisms}: $\hom[\Rel]{X}{Y} = \{R \mid R \subseteq X \times Y\}$ relations between $X$ and $Y$.
\item \textbf{Identities}: $\id[X] = \{(x, x) \mid x \in X\}$
\item \textbf{Composition}: if $f : X \to Y$ and $g : Y \to Z$, then $f; g = \{ (x, z) \mid \exists y.\, (x, y) \in f \land (y, z) \in g\}$.
\end{itemize}

Since identities and composition are defined just as in \Set, we don't have any new proofs to do.

\subsection{Types}
\label{sec:types}

The category \Type consists of:
\begin{itemize}
\item \textbf{Objects}: Types in extended STLC (STLC + unit, products, and sums)
\item \textbf{Morphisms}: $\hom[\Type]{\tau}{\sigma} = \{ [e]_{\cong_{\alpha,\beta\eta}} \mid \exists x.\, x : \tau  \vdash e : \sigma\}$ $\alpha,\beta,\eta$-equivalence classes of terms with a single free variable.
\item \textbf{Identities}: $\id[\tau] = [x]_{\cong_{\alpha,\beta,\eta}}$.
\item \textbf{Composition}: let $e$ and $e'$ be such that $x : \tau \vdash e : \sigma$ and $y : \sigma \vdash e' : \rho$.
  Then $e; e' = e'[y \mapsto e]$.
\end{itemize}

Composition with the identity on the left is then $e[x \mapsto x] = e$, while composition on the right is $x[x \mapsto e] = e$.
$$
\begin{array}{ll}
  e_1; (e_2; e_3 ) &= (e_3 [y \mapsto e_2]) [x \mapsto e_1]\\
                   &= (e_3[y \mapsto e_2 [x \mapsto e_1]]~~~\text{(because $x \notin \text{fv}(e_3)$)}\\
                   &=(e_1; e_2); e_3
\end{array}
$$

\section{Opposite Categories}
\label{sec:opposite-categories}

\begin{itemize}
\item For every category~$\mathcal{C}$ there is an ``opposite'' category $\op{\mathcal{C}}$ where all of the morphisms are flipped.
\item More precisely, we get the following:
  \begin{defn}[Opposite Category]
    For a category~$\mathcal{C}$, the category~$\op{\mathcal{C}}$ has:
    \begin{itemize}
    \item \textbf{Objects}: The objects of $\mathcal{C}$
    \item \textbf{Morphisms}: $\hom[\op{\mathcal{C}}]{X}{Y} = \hom[\mathcal{C}]{Y}{X}$
    \item \textbf{Identities}: As in $\mathcal{C}$
    \item \textbf{Composition}: Reversed from $\mathcal{C}$: $f;g$ in $\op{\mathcal{C}}$ is $g; f$ in $\mathcal{C}$.
    \end{itemize}
    Since identities and composition are defined in $\mathcal{C}$, we don't have to prove anything more.
  \end{defn}
\item This is a very simple definition, but it will turn out to be incredibly valuable.
\item In particular, we will see next time that the notion of ``duality'' that we talked about when defining type constructions can be formalized via opposite categories.
\end{itemize}


\end{document}

%%% Local Variables:
%%% mode: latex
%%% TeX-master: t
%%% TeX-engine: luatex
%%% End:
